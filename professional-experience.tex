\begin{cvlist}{Professional experience}
  % Nov 2015 - current
  \item[\footnotesize{Nov 2015 - current}] \textbf{Software Development Engineer II at Amazon}\\
  \href{http://www.amazon.com}{Amazon}\hfill\textit{Dublin, IE}\\
  Additional roles: Tech leader, Solution Architect, Scrum Master
  \begin{small}
    \begin{itemize}
      \item Designed and implemented asynchronous APIs to manage the network monitoring micro-services.
      \item Designed and implemented the orchestration system that controls and distributes workload to the network micro-services.
    \end{itemize}
  \end{small}
  % Jan 14 2013 - Oct 2015
  \item[\footnotesize{Jan 2013 - Oct 2015}] \textbf{Software Development Engineer I at Amazon}\\
  \href{http://www.amazon.com}{Amazon}\hfill\textit{Dublin, IE}
  \begin{small}
    \begin{itemize}
      \item Designed and developed an asynchronous event based and highly scalable network monitoring system.
    \end{itemize}
  \end{small}
  % Aug-Sept 27 jun - 30 oct 2012
  \item[\footnotesize{Aug 2012 - Oct 2012}] \textbf{Software Developer in Django/Python}\\
  \href{http://opmservice.com}{Orloff Property Management}\hfill\textit{Berkeley, CA}
  \begin{small}
    \begin{itemize}
      \item Full stack web application for the management of
            thousands of properties located in the states
            of California and Indiana.\\
\ifthenelse{\boolean{extended}}{
            My tasks were to develop CRUD operations and
            more additional features such as generation of
            reports and filter/sort operations.
            The technology used was Django as Web Framework,
            JQuery to manage asynchronous request to the server.
            The deployment system used is Fabric in order
            to manage complex actions and ensure that the code
            has correctly deployed into the server.
            For testing the code it was
            used Selenium IDE and Selenium WebDriver
            test automation tool.
}
    \end{itemize}
  \end{small}
  % Sept 2011 - Feb 2012
  \item[\footnotesize{Sep 2011 - Feb 2012}] \textbf{Researcher High Performance Computing}\\
  \href{http://www.inco2.upv.es/menu.php}{Interdisciplinary Computation and Communication Group}\hfill\textit{Valencia, ES}
  \begin{small}
    \begin{itemize}
      \item Optimized the computation of algorithms for solving
            the eigenvalue problem via graphics cards using CUDA.
\ifthenelse{\boolean{extended}}{
            There are only few libraries that support
            the development of advanced algorithms in numerical
            linear algebra
            and some of them implement the main routines
            for linear algebra. It
            was necessary to develop projects to aim
            the improvement and
            integration of the existing libraries for graphics card
            programming.
            The Unit testing activity represented the main
            approach for the stability and precision of
            the algorithms.
            The Interdisciplinary Computation and
            Communication Group (INCO2-DSIC)
            is made up of researchers belonging to the Department of
            Information Systems and Computation (DSIC)
            of the Universidad
            Politécnica de Valencia.
}
    \end{itemize}
  \end{small}
  %May 18 2009-- Aug 17 2009.
  \item[\footnotesize{May 2009 - Aug 2009}] \textbf{Java Analyst and Programmer}\\
  \href{http://www.zenitlab.it/}{ZENIT Information Systems}\hfill\textit{Cosenza, IT}
  \begin{small}
    \begin{itemize}
      \item Designed and developed the desktop client-side
            application for an Information System built
            via OpenSwing library and used Java EJB.
\ifthenelse{\boolean{extended}}{
            The customer, Pubbliemme, is a leader for concession
            of advertising space in the center-south of Italy
            with over 2500 advertising establishments.
}
    \end{itemize}
  \end{small}
\end{cvlist}