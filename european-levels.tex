

{\large \scshape \hypertarget{eur}European levels - Self assessment grid\footnote{Common European Framework of Reference for Languages}}\\

\vspace{0.5em}

\begin{tabular}{|c|p{1.5cm}|p{2cm}|p{2cm}|p{2cm}|p{2cm}|p{2cm}|p{2cm}|} \hline
& & \textbf{A1} & \textbf{A2} & \textbf{B1} & \textbf{B2} & \textbf{C1} & \textbf{C2} \\ \hline

%\multirow{2}{*}{\rotatebox{90}{text}} &
%\multirow{2}{15mm}{\begin{sideways}\parbox{15mm}{\scshape understanding}\end{sideways}} &
\multirow{2}{*}{\begin{sideways}\Large\scshape understanding\end{sideways}} &
\small\scshape Listening &
\scriptsize I can understand
familiar words and
very basic phrases
concerning myself,
my family and
immediate
concrete
surroundings when
people speak
slowly and clearly. &

\scriptsize I can understand
phrases and the
highest frequency
vocabulary related
to areas of most
immediate personal
relevance (e.g.
very basic personal
and family
information,
shopping, local
area, employment).
I can catch the
main point in short,
clear, simple
messages and
announcements. &

\scriptsize I can understand
the main points of
clear standard
speech on familiar
matters regularly
encountered in
work, school,
leisure, etc. I can
understand the
main point of many
radio or TV
programmes on
current affairs or
topics of personal
or professional
interest when the
delivery is
relatively slow and
clear. &

\scriptsize I can understand
extended speech
and lectures and
follow even
complex lines of
argument provided
the topic is
reasonably
familiar. I can
understand most
TV news and
current affairs
programmes. I can
understand the
majority of films in
standard dialects. &

\scriptsize I can understand
extended speech
even when it is not
clearly structured
and when
relationships are
only implied and
not signalled
explicitly. I can
understand
television
programmes and
films without too
much effort. &

\scriptsize I have no difficulty
in understanding
any kind of spoken
language, whether
live or broadcast,
even when
delivered at fast
native speed,
provided. I have
some time to get
familiar with the
accent. \\ \cline{2-8}


& \small\scshape Reading &

\scriptsize I can understand
familiar names,
words and very
simple sentences,
for example on
notices and posters
or in catalogues. &

\scriptsize I can read very
short, simple texts.
I can find specific,
predictable
information in
simple everyday
material such as
advertisements,
prospectuses,
menus and
timetables and I
can understand
short simple
personal letters. &

\scriptsize I can understand
texts that consist
mainly of high
frequency
everyday or
job-related
language. I can
understand the
description of
events, feelings
and wishes in
personal letters. &

\scriptsize I can read articles
and reports
concerned with
contemporary
problems in which
the writers adopt
particular attitudes
or viewpoints. I
can understand
contemporary
literary prose. &

\scriptsize I can understand
long and complex
factual and literary
texts, appreciating
distinctions of
style. I can
understand
specialised articles
and longer
technical
instructions, even
when they do not
relate to my field. &

\scriptsize I can read with
ease virtually all
forms of the written
language, including
abstract,
structurally or
linguistically
complex texts such
as manuals,
specialised articles
and literary works. \\ \hline

\multirow{2}{*}{\begin{sideways}\Large\scshape speaking\end{sideways}} &

\small\scshape Spoken interaction &

\scriptsize I can interact in a
simple way
provided the other
person is prepared
to repeat or
rephrase things at
a slower rate of
speech and help
me formulate what
I'm trying to say. I
can ask and
answer simple
questions in areas
of immediate need
or on very familiar
topics. &

\scriptsize I can communicate
in simple and
routine tasks
requiring a simple
and direct
exchange of
information on
familiar topics and
activities. I can
handle very short
social exchanges,
even though I can't
usually understand
enough to keep the
conversation going
myself. &

\scriptsize I can deal with
most situations
likely to arise
whilst travelling in
an area where the
language is
spoken. I can enter
unprepared into
conversation on
topics that are
familiar, of
personal interest
or pertinent to
everyday life (e.g.
family, hobbies,
work, travel and
current events). &

\scriptsize I can interact with
a degree of fluency
and spontaneity
that makes regular
interaction with
native speakers
quite possible. I
can take an active
part in discussion
in familiar
contexts,
accounting for and
sustaining my
views. &

\scriptsize I can express
myself fluently and
spontaneously
without much
obvious searching
for expressions. I
can use language
flexibly and
effectively for
social and
professional
purposes. I can
formulate ideas
and opinions with
precision and relate
my contribution
skilfully to those of
other speakers. &

\scriptsize I can take part
effortlessly in any
conversation or
discussion and
have a good
familiarity with
idiomatic
expressions and
colloquialisms. I
can express myself
fluently and convey
finer shades of
meaning precisely.
If I do have a
problem I can
backtrack and
restructure around
the difficulty so
smoothly that other
people are hardly
aware of it. \\ \cline{2-8}

& \small\scshape Spoken production &

\scriptsize I can use simple
phrases and
sentences to
describe where I
live and people I
know. &

\scriptsize I can use a series
of phrases and
sentences to
describe in simple
terms my family
and other people,
living conditions,
my educational
background and my
present or most
recent job. &

\scriptsize I can connect
phrases in a simple
way in order to
describe
experiences and
events, my
dreams, hopes and
ambitions. I can
briefly give reasons
and explanations
for opinions and
plans. I can
narrate a story or
relate the plot of a
book or film and
describe my
reactions. &

\scriptsize I can present clear,
detailed
descriptions on a
wide range of
subjects related to
my field of interest.
I can explain a
viewpoint on a
topical issue giving
the advantages
and disadvantages
of various options. &

\scriptsize I can present clear,
detailed
descriptions of
complex subjects
integrating
subthemes,
developing
particular points
and rounding off
with an appropriate
conclusion. &

\scriptsize I can present a
clear,
smoothly-flowing
description or
argument in a style
appropriate to the
context and with an
effective logical
structure which
helps the recipient
to notice and
remember
significant points. \\ \hline

\end{tabular}

\pagebreak

\begin{tabular}{|c|p{1.5cm}|p{2cm}|p{2cm}|p{2cm}|p{2cm}|p{2cm}|p{2cm}|} \hline
& & \textbf{A1} & \textbf{A2} & \textbf{B1} & \textbf{B2} & \textbf{C1} & \textbf{C2} \\ \hline

\multirow{1}{*}{\begin{sideways}\Large\scshape writing\end{sideways}} &

\small\scshape Writing &

\scriptsize I can write a short,
simple postcard,
for example
sending holiday
greetings. I can fill
in forms with
personal details,
for example
entering my name,
nationality and
address on a hotel
registration form. &

\scriptsize I can write short,
simple notes and
messages. I can
write a very simple
personal letter, for
example thanking
someone for
something. &

\scriptsize I can write simple
connected text on
topics which are
familiar or of
personal interest. I
can write personal
letters describing
experiences and
impressions. &

\scriptsize I can write clear,
detailed text on a
wide range of
subjects related to
my interests. I can
write an essay or
report, passing on
information or
giving reasons in
support of or
against a particular
point of view. I can
write letters
highlighting the
personal
significance of
events and
experiences. &

\scriptsize I can express
myself in clear,
wellstructured text,
expressing points
of view at some
length. I can write
about complex
subjects in a letter,
an essay or a
report, underlining
what I consider to
be the salient
issues. I can select
a style appropriate
to the reader in
mind. &

\scriptsize I can write clear,
smoothly-flowing
text in an
appropriate style. I
can write complex
letters, reports or
articles which
present a case with
an effective logical
structure which
helps the recipient
to notice and
remember
significant points. I
can write
summaries and
reviews of
professional or
literary works. \\ \hline

\end{tabular}